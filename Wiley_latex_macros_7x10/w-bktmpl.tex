%%%%%%%%%%%%%%
%% Run LaTeX on this file several times to get Table of Contents,
%% cross-references, and citations.

%% w-bktmpl.tex. Current Version: Feb 16, 2012
%%%%%%%%%%%%%%%%%%%%%%%%%%%%%%%%%%%%%%%%%%%%%%%%%%%%%%%%%%%%%%%%
%
%  Template file for
%  Wiley Book Style, Design No.: SD 001B, 7x10
%  Wiley Book Style, Design No.: SD 004B, 6x9
%
%  Prepared by Amy Hendrickson, TeXnology Inc.
%  http://www.texnology.com
%%%%%%%%%%%%%%%%%%%%%%%%%%%%%%%%%%%%%%%%%%%%%%%%%%%%%%%%%%%%%%%%

%%%%%%%%%%%%%%%%%%%%%%%%%%%%%%%%%%%%%%%%%%%%%%%%%%%%%%%%%%%%%%%%
%% Class File

%% For default 7 x 10 trim size:
\documentclass{WileySev}

%% Or, for 6 x 9 trim size
%\documentclass{WileySix}

%%%%%%%%%%%%%%%%%%%%%%%%%%%%%%%%%%%%%%%%%%%%%%%%%%%%%%%%%%%%%%%%
%% Post Script Font File

% For PostScript text
% If you have font problems, you may edit the w-bookps.sty file
% to customize the font names to match those on your system.
\usepackage{w-bookps}
\usepackage[a4paper]{geometry}
\usepackage{parskip}
\usepackage{zhnumber}
\usepackage{hyperref}
\hypersetup{hidelinks}


%首行缩进2个字符
\usepackage{indentfirst}
\setlength{\parindent}{2em}
%设置段落间距
\setlength{\parskip}{0.5em}

% 采用临时办法来设置奇偶页的边距,留出装订线
% TODO: 应该使用 geometry 的 twoside
\setlength{\evensidemargin}{-36pt}

%%%%%%%
%% For times math: However, this package disables bold math (!)
%% \mathbf{x} will still work, but you will not have bold math
%% in section heads or chapter titles. If you don't use math
%% in those environments, mathptmx might be a good choice.

% \usepackage{mathptmx}


%%%%%%%%%%%%%%%%%%%%%%%%%%%%%%%%%%%%%%%%%%%%%%%%%%%%%%%%%%%%%%%%
%% Graphicx.sty for Including PostScript .eps files

\usepackage{graphicx}

%%%%%%%%%%%%%%%%%%%%%%%%%%%%%%%%%%%%%%%%%%%%%%%%%%%%%%%%%%%%%%%%
%% Other packages you might want to use:

% for chapter bibliography made with BibTeX
% \usepackage{chapterbib}

% for multiple indices
% \usepackage{multind}

% for answers to problems
% \usepackage{answers}

%%%%%%%%%%%%%%%%%%%%%%%%%%%%%%%%%%%%%%%%%%%%%%%%%%%%%%%%%%%%%%%%
%% Change options here if you want:
%%
%% How many levels of section head would you like numbered?
%% 0= no section numbers, 1= section, 2= subsection, 3= subsubsection
%%==>>
\setcounter{secnumdepth}{3}

%% How many levels of section head would you like to appear in the
%% Table of Contents?
%% 0= chapter titles, 1= section titles, 2= subsection titles, 
%% 3= subsubsection titles.
%%==>>
\setcounter{tocdepth}{2}

%% Cropmarks? good for final page makeup
%% \docropmarks %% turn cropmarks on

%%%%%%%%%%%%%%%%%%%%%%%%%%%%%%%%%%%%%%%%%%%%%%%%%%%%%%%%%%%%%%%%
%% DRAFT
%
% Uncomment to get double spacing between lines, current date and time
% printed at bottom of page.
% \draft
% (If you want to keep tables from becoming double spaced also uncomment
% this):
% \renewcommand{\arraystretch}{0.6}
%%%%%%%%%%%%%%%%%%%%%%%%%%%%%%

\begin{document}

%%%%%%%%%%%%%%%%%%%%%%%%%%%%%%%%%%%%%%%%%%%%%%%%%%%%%%%%%%%%%%%%
%% Title Pages
%%
%% Wiley will provide title and copyright page, but you can make
%% your own titlepages if you'd like anyway

%% Setting up title pages, type in the appropriate names here:
\booktitle{Wiley 书籍中文模板}
\subtitle{非 Wiley 官方中文书籍模板}

\author{CookWhy@163.com}
%or
\authors{}

%% \\ will start a new line.
%% You may add \affil{} for affiliation, ie,
%\authors{Robert M. Groves\\
%\affil{Universitat de les Illes Balears}
%Floyd J. Fowler, Jr.\\
%\affil{University of New Mexico}
%}

%% Print Half Title and Title Page:
\halftitlepage
\titlepage


%%%%%%%%%%%%%%%%%%%%%%%%%%%%%%%%%%%%%%%%%%%%%%%%%%%%%%%%%%%%%%%%
%% Off Print Info

%% Add your info here:
\offprintinfo{title, edition}{author}

%% Can use \\ if title, and edition are too wide, ie,
%% \offprintinfo{Survey Methodology,\\ Second Edition}{Robert M. Groves}


%%%%%%%%%%%%%%%%%%%%%%%%%%%%%%%%%%%%%%%%%%%%%%%%%%%%%%%%%%%%%%%%
%% Copyright Page

\begin{copyrightpage}{year}
Title, etc
\end{copyrightpage}

% Note, you must use \ to start indented lines, ie,
% 
% \begin{copyrightpage}{2004}
% Survey Methodology / Robert M. Groves . . . [et al.].
% \       p. cm.---(Wiley series in survey methodology)
% \    ``Wiley-Interscience."
% \    Includes bibliographical references and index.
% \    ISBN 0-471-48348-6 (pbk.)
% \    1. Surveys---Methodology.  2. Social 
% \  sciences---Research---Statistical methods.  I. Groves, Robert M.  II. %
% Series.\\

% HA31.2.S873 2004
% 001.4'33---dc22                                             2004044064
% \end{copyrightpage}

%%%%%%%%%%%%%%%%%%%%%%%%%%%%%%%%%%%%%%%%%%%%%%%%%%%%%%%%%%%%%%%%
%% Frontmatter >>>>>>>>>>>>>>>>

%%%%%%%%%%%%%%%%%%%%%%%%%%%%%%%%%%%%%%%%%%%%%%%%%%%%%%%%%%%%%%%%
%% Only Dedication (optional) 
%% or Contributor Page for edited books
%% before \tableofcontents

\dedication{感谢 Wiley 提供如此好的英文模板,基于此模板进行中文书籍模板制作,显得非常轻松与美观}

% ie,
%\dedication{To my parents}

%%%%%%%%%%%%%%%%%%%%%%%%%%%%%%%%%%%%%%%%%%%%%%%%%%%%%%%%%%%%%%%%
%  Contributors Page for Edited Book
%%%%%%%%%%%%%%%%%%%%%%%%%%%%%%%%%%%%%%%%%%%%%%%%%%%%%%%%%%%%%%%%

% If your book has chapters written by different authors,
% you'll need a Contributors page.

% Use \begin{contributors}...\end{contributors} and
% then enter each author with the \name{} command, followed
% by the affiliation information.

% \begin{contributors}
% \name{Masayki Abe,} Fujitsu Laboratories Ltd., Fujitsu Limited, Atsugi,
% Japan

% \name{L. A. Akers,} Center for Solid State Electronics Research, Arizona
% State University, Tempe, Arizona

% \name{G. H. Bernstein,} Department of Electrical and
% Computer Engineering, University of Notre Dame, Notre Dame, South Bend, 
% Indiana; formerly of
% Center for Solid State Electronics Research, Arizona
% State University, Tempe, Arizona 
% \end{contributors}

%%%%%%%%%%%%%%%%%%%%%%%%%%%%%%%%%%%%%%%%%%%%%%%%%%%%%%%%%%%%%%%%
% 简短目录,可选
% 只到目录的第一级
\contentsinbrief %optional
\tableofcontents
% \listoffigures %optional
% \listoftables  %optional

%%%%%%%%%%%%%%%%%%%%%%%%%%%%%%%%%%%%%%%%%%%%%%%%%%%%%%%%%%%%%%%%
% Optional Foreword:
% 序言,可由他人作序,或自序

\begin{foreword}
这里是前言的样例,可以置于 forword 命令里。

象牙塔里的绅士总会假清高的笑骂:“政治家,政治家,你算得什么艺术家呢!你的艺术是有倾向的!”对于这种嘲笑,革命文学家只有一个回答:

你想用什么来骂倒我呢?难道因为我要改造世界的那种热诚的巨大火焰, 它在我的艺术里也在燃烧着么?

\prefaceauthor{卢纳察尔斯基}
\where{海淀区花园桥, 北京\\
2021年9月}
\end{foreword}

%%%%%%%%%%%%%%%%%%%%%%%%%%%%%%%%%%%%%%%%%%%%%%%%%%%%%%%%%%%%%%%%
% Optional Preface:
% 前言

\begin{preface}
一本书的附件包括书前部分的序、前言、致谢、题词、编撰者名单、作者简介、附录等等。
不是每本书都有所有的附件,取舍全看作者的选择。其中容易搞混的是Foreword 和Preface。中文都可以翻译成序言或前言。在英文里,区别是前者由他人撰写,后者由作者自己写。一本书的Foreword 可以有几个,分别请不同的名人或同行撰写。而每一版的Preface 只有一个。如果是重版书,新版里可以把之前旧版的所有前言罗列,为的是方便作者了解该书的新旧更替。一本书的出版,经常不仅仅是作者一个人或几个人的努力,致谢部分是最好的用武之地,可以让作者答谢所有为之付出努力的团队。

以上诸多细则里,涉及公式、参考文献和使用许可的问题最为普遍,也是出版过程中最掣肘的因素之一。
\prefaceauthor{作者}
\where{广州天河\\
2021年9月}
\end{preface}

% ie,
% \begin{preface}
% This is an example preface.
% \prefaceauthor{R. K. Watts}
% \where{Durham, North Carolina\\
% September, 2004}

%%%%%%%%%%%%%%%%%%%%%%%%%%%%%%%%%%%%%%%%%%%%%%%%%%%%%%%%%%%%%%%%
% Optional Acknowledgments:

% \acknowledgments
% acknowledgment text
% \authorinitials{} % ie, I. R. S.


%%%%%%%%%%%%%%%%%%%%%%%%%%%%%%%%
%% Glossary Type of Environment:

% \begin{glossary}
% \term{<term>}{<description>}
% \end{glossary}

%%%%%%%%%%%%%%%%%%%%%%%%%%%%%%%%
% \begin{acronyms} 
% \acro{<term>}{<description>}
% \end{acronyms}

%%%%%%%%%%%%%%%%%%%%%%%%%%%%%%%%
%% In symbols environment <term> is expected to be in math mode; 
%% if not in math mode, use \term{\hbox{<term>}}

% \begin{symbols}
% \term{<math term>}{<description>}
% \term{\hbox{<non math term>}}Box used when not using a math symbol.
% \end{symbols}

%%%%%%%%%%%%%%%%%%%%%%%%%%%%%%%%
% \begin{introduction}
%\introauthor{<name>}{<affil>}
% Introduction text...
% \end{introduction}

%%%%%%%%%%%%%%%%%%%%%%%%%%%%%%%%%%%%%%%%%%%%%%%%%%%%%%%%%%%%%%%%
%% End for Front Matter, Beginning of text of book  >>>>>>>>>>>

%% Short version of title without \\ may be written in sq. brackets:

%% Optional Part :
\part[简单说明]
{简单说明\\ 太长可换行}

\chapter[章节标题不带名人名言]
{章节标题不带名人名言 \\ 太长可换行}

本模板基于 Wiley 英文模板制作,原模板名为 \textit{The Wiley Book Style},网址为:

\url{https://www.latextemplates.com/template/wiley-book-style}

Wiley出版社1807年创建于美国,是一家具有超过200年历史的全球知名的出版机构,面向专业人士、科研人员、教育工作者、学生、终身学习者提供必需的知识和服务。
Wiley是一家提供必备内容解决方案的全球供应商,旨在提高科研、 教育、专业实践行为的产出。Wiley及旗下的子品牌出版了超过500位诺贝尔奖得主的作品。

本模板保留了 Wiley 的版本信息及 Logo 等,并把纸张尺寸根据中国的习惯,改为 A4,使用者可按需修改这类信息。

本模板未取得 Wiley 授权,请使用者自行确认版权信息。

%%%%%%%%%%%%%%%%%%%%%%%%%%%%%%%%%%%%%%%%%%%%%%%%%%%%%%
%% optional prologue or prologues
% \chapter{Chapter Title}
% \prologue{<text>}{<author attribution>}

\chapter[章节标题带名人名言]
{章节标题带名人名言 \\ 太长可换行}
% 名人名言可用 prologue 置于 chapter 目录下
\prologue{优秀的人讨论思想,普通的人讨论事件,狭隘的人讨论人。}{埃莉诺·罗斯福,罗斯福总统夫人}

本模板主要把 Wiley 模板的 CHAPTER、PART 等关键词翻译成中文并设置好版面格式。

版权声明:本模板纯粹出于个人爱好进行汉化,并未得到 Wiley 授权,如果需要使用此模板并对版权有顾虑,请联系 Wiley 进行商讨。

Wiley 原始模板地址:https://www.latextemplates.com/template/wiley-book-style

主要修改内容:

1. 章节名称的汉化。

2. 页面大小改为 A4 大小,更加符合中国人习惯;原版为 6x9 或是 7x10 英寸,原版更符合美国人习惯。

3. 对中文字符集进行了设置。

4. 对奇偶页装订线进行了修改,更符合书籍装订习惯。

如果有修改需求,请在站内留言。

\section{待修改内容}

1. 中文奇偶页边距设置,用更好的方式来设置。

2. 中英文混排字体优化。

3. 尽可能使用 CTeX 来进行章节名称汉化,现在是直接在原来的 cls 文件里手工修改,维护十分不方便。

\section{样例一:岳阳楼记}

庆历四年春,滕子京谪守巴陵郡。越明年,政通人和,百废具兴。乃重修岳阳楼,增其旧制,刻唐贤今人诗赋于其上;属予作文以记之。

予观夫巴陵胜状,在洞庭一湖,衔远山,吞长江,浩浩汤汤,横无际涯,朝晖夕阴,气象万千,此则岳阳楼之大观也;前人之述备矣。然则北通巫峡,南极潇湘,迁客骚人,多会于此,览物之情,得无异乎?

若夫霪雨霏霏,连月不开,阴风怒号,浊浪排空,日星隐曜,山岳潜形;商旅不行,樯倾楫摧;薄暮冥冥,虎啸猿啼;登斯楼也,则有去国怀乡,忧谗畏讥,满目萧然,感极而悲者矣!

至若春和景明,波澜不惊,上下天光,一碧万顷,沙鸥翔集,锦鳞游泳,岸芷汀兰,郁郁青青;而或长烟一空,皓月千里,浮光耀金,静影沈璧,渔歌互答,此乐何极!登斯楼也,则有心旷神怡,宠辱皆忘,把酒临风,其喜洋洋者矣!

嗟夫!予尝求古仁人之心,或异二者之为,何哉?不以物喜,不以己悲。居庙堂之高,则忧其民,处江湖之远,则忧其君,是进亦忧,退亦忧。然则何时而乐耶?其必曰「先天下之忧而忧,后天下之乐而乐」欤!噫,微斯人,吾谁与归!

\section{Here is a normal section}
Here is some text.

\subsection{This is the subsection}
Here is some normal text.
Here is some normal text.
Here is some normal text.
Here is some normal text.
Here is some normal text.
Here is some normal text.
Here is some normal text.
Here is some normal text.
Here is some normal text.
Here is some normal text.
Here is some normal text.

%%%%%%%%%%%%%%%%%%%%%%%%%%%%%%%%%%%%%%%%%%%%%
% Edited Book: Author and Affiliation
%%%%%%%%%%%%%%%%%%%%%%%%%%%%%%%%%%%%%%%%%%%%%

% After \chapter{Chapter Title}, you can
% enter the author name and embed the affiliation with
% \chapterauthors{(author name, or names)
% \chapteraffil{(affiliation or affiliations)}
% }    

% For instance:
% \chapter{Chapter Title}
% \chapterauthors{G. Alvarez and R. K. Watts
% \chapteraffil{Carnegie Mellon University, Pittsburgh, Pennsylvania}

% For separate affiliations you can use \affilmark{(number)} after
% the name of a particular author and before the matching affiliation:

% For instance:
% \chapter{Chapter Title}
% \chapterauthors{George Smeal, Ph.D.\affilmark{1}, Sally Smith,
% M.D.\affilmark{2}, and Stanley Kubrick\affilmark{1}
% \chapteraffil{\affilmark{1}AT\&T Bell Laboratories
% Murray Hill, New Jersey\\
% \affilmark{2}Harvard Medical School,
% Boston, Massachusetts}
% }

\part[更多排版样式说明]
{更多排版样式说明\\ 太长可换行}

\chapter[排版工具说明与举例]
{排版工具说明与举例}
\prologue{工欲善其事,必先利其器。}{《论语·卫灵公》}

参考文献,可正常使用 BibTeX 格式即可。

一切格式,可参考本源文件格式进行;Wiley 原始文件包的文件名、结构均没有修改。

更多排版问题可在本项目的 Github 上留言。

%%%%%%%%%%%%%%%%%%%%%%%

%% short version of section head, or one without \\ supplied in sq. brackets.

% \section[Introduction and fugue]{Introduction\\ and fugue}
% \subsection[This is the subsection]{This is the\\ subsection}
% \subsubsection{This is the subsubsection}
% \paragraph{This is the paragraph}

% \begin{chapreferences}{widest label}
% \bibitem{<label>}Reference
% \end{chapreferences}

% optional chapter bibliography using BibTeX,
% must also have \usepackage{chapterbib} before \begin{document}
% Must use root file with \include{chap1}, \include{chap2} form.
%\bibliographystyle{plain}
%\bibliography{<your .bib file name>}

% optional appendix at the end of a chapter:
% \chapappendix{<chap appendix title>}
% \chapappendix{} % no title

%%%%%%%%%%%%%%%%%%%%%%%%%%%%%%%%%%%%%%%%%%%%%%%%%%%%%%%%%%%%%%%%
%% End Matter >>>>>>>>>>>>>>>>>>

% \appendix{<optional title for appendix at end of book>}
% \appendix{} % appendix without title

% \begin{references}{<widest label>}
% \bibitem{sampref}Here is reference.
% \end{references}

%%%%%%%%%%%%%%%%%%%%%%%%%%%%%%%%%%%%%%%%%%%%%%%%%%%%%%%%%%%%%%%%
%% Optional Problem Sets: Can use this at the end of each chapter or at end
%% of book

% \begin{problems}
% \prob
% text

% \prob
% text

% \subprob
% text

% \subprob
% text

% \prob
% text
% \end{problems}

%%%%%%%%%%%%%%%%%%%%%%%%%%%%%%%%%%%%%%%%%%%%%%%%%%%%%%%%%%%%%%%%
%% Optional Exercises: Can use this at the end of each chapter or at end
%% of book

% \begin{exercises}
% \exer
% text

% \exer
% text

% \subexer
% text

% \subexer
% text

% \exer
% text
% \end{exercises}


%%%%%%%%%%%%%%%%%%%%%%%%%%%%%%%%%%%%%%%%%%%%%%%%%%%%%%%%%%%%%%%%
%% INDEX: Use only one index command set:

%% 1) The default LaTeX Index
\printindex

%% 2) For Topic index and Author index:

% \usepackage{multind}
% \makeindex{topic}
% \makeindex{authors}
% \begin{document}
% ...
% add index terms to your book, ie,
% \index{topic}{A term to go to the topic index}
% \index{authors}{Put this author in the author index}

%% (these are Wiley commands)
%\multiprintindex{topic}{Topic index}
%\multiprintindex{authors}{Author index}

\end{document}

%%%%%%% Demo of section head containing sample macro:
%% To get a macro to expand correctly in a section head, with upper and
%% lower case math, put the definition and set the box 
%% before \begin{document}, so that when it appears in the 
%% table of contents it will also work:

\newcommand{\VT}[1]{\ensuremath{{V_{T#1}}}}

%% use a box to expand the macro before we put it into the section head:

\newbox\sectsavebox
\setbox\sectsavebox=\hbox{\boldmath\VT{xyz}}

%%%%%%%%%%%%%%%%% End Demo


Other commands, and notes on usage:

-----
Possible section head levels:
\section{Introduction}
\subsection{This is subsection}
\subsubsection{This is subsubsection}
\paragraph{This is the paragraph}

-----
Tables:
 Remember to use \centering for a small table and to start the table
 with \hline, use \hline underneath the column headers and at the end of 
 the table, i.e.,

\begin{table}[h]
\caption{Small Table}
\centering
\begin{tabular}{ccc}
\hline
one&two&three\\
\hline
C&D&E\\
\hline
\end{tabular}
\end{table}

For a table that expands to the width of the page, write

\begin{table}
\begin{tabular*}{\textwidth}{@{\extracolsep{\fill}}lcc}
\hline
....
\end{tabular*}
%% Sample table notes:
\begin{tablenotes}
$^a$Refs.~19 and 20.

$^b\kappa, \lambda>1$.
\end{tablenotes}
\end{table}

-----
Algorithm.
Maintains same fonts as text (as opposed to verbatim which uses fixed
width fonts). Space at beginning of line will be maintained if you
use \ at beginning of line.

\begin{algorithm}
{\bf state\_transition algorithm} $\{$
\        for each neuron $j\in\{0,1,\ldots,M-1\}$
\        $\{$   
\            calculate the weighted sum $S_j$ using Eq. (6);
\            if ($S_j>t_j$)
\                    $\{$turn ON neuron; $Y_1=+1\}$   
\            else if ($S_j<t_j$)
\                    $\{$turn OFF neuron; $Y_1=-1\}$   
\            else
\                    $\{$no change in neuron state; $y_j$ remains %
unchanged;$\}$ .
\        $\}$   
$\}$   
\end{algorithm}

-----
Sample quote:
\begin{quote}
quotation...
\end{quote}

-----
Listing samples

\begin{enumerate}
\item
This is the first item in the numbered list.

\item
This is the second item in the numbered list.
\end{enumerate}

\begin{itemize}
\item
This is the first item in the itemized list.

\item
This is the first item in the itemized list.
This is the first item in the itemized list.
This is the first item in the itemized list.
\end{itemize}

\begin{itemize}
\item[]
This is the first item in the itemized list.

\item[]
This is the first item in the itemized list.
This is the first item in the itemized list.
This is the first item in the itemized list.
\end{itemize}

%% Index commands
Author and Topic Indices, See docs.pdf and w-bksamp.pdf
